\documentclass[lang=cn,newtx,10pt,scheme=chinese]{elegantbook}
\usepackage{realboxes}

\title{ESP32教程}
\author{左元}

\setcounter{tocdepth}{3}

\cover{cover.pdf}

% 本文档命令
\usepackage{array}
\newcommand{\ccr}[1]{\makecell{{\color{#1}\rule{1cm}{1cm}}}}

% 修改标题页的橙色带
\definecolor{customcolor}{RGB}{32,178,170}
\colorlet{coverlinecolor}{customcolor}
\usepackage{cprotect}

\newtcolorbox{marker}[1][]{enhanced,
  before skip=2mm,after skip=3mm,
  boxrule=0.4pt,left=5mm,right=2mm,top=1mm,bottom=1mm,
  colback=yellow!50,
  colframe=yellow!20!black,
  sharp corners,rounded corners=southeast,arc is angular,arc=3mm,
  underlay={%
    \path[fill=tcbcolback!80!black] ([yshift=3mm]interior.south east)--++(-0.4,-0.1)--++(0.1,-0.2);
    \path[draw=tcbcolframe,shorten <=-0.05mm,shorten >=-0.05mm] ([yshift=3mm]interior.south east)--++(-0.4,-0.1)--++(0.1,-0.2);
    \path[fill=yellow!50!black,draw=none] (interior.south west) rectangle node[white]{\Huge\bfseries !} ([xshift=4mm]interior.north west);
    },
  drop fuzzy shadow,#1}

\tcbuselibrary{listings, skins, breakable}
\usepackage[T1]{fontenc}
\usepackage[ttdefault=true]{AnonymousPro}
\definecolor{pblue}{rgb}{0.13,0.13,1}
\definecolor{pgreen}{rgb}{0,0.5,0}

\newtcblisting[auto counter, number within=chapter]{mycode}[1]{
    breakable,
    enhanced,
    attach boxed title to top right={yshift=-\tcboxedtitleheight},
    boxed title style={
        size=small,colback=gray!50,
        colframe=gray!50,
        sharp corners=downhill,
        arc=.5cm,
        top=1mm,bottom=1mm,left=1mm,right=1mm
    },
    fonttitle=\color{black}\itshape\ttfamily,
    colframe=gray!20,
    top=\tcboxedtitleheight,
    bottom=\tcboxedtitleheight,
    sharp corners=downhill,
    arc=.5cm,
    title={#1},
    listing only,
    listing options={
        escapeinside={(*@}{@*)},
        language=c,
        basicstyle=\fontfamily{AnonymousPro}\selectfont,
        keywordstyle=\bfseries\color{pblue},
        stringstyle=\bfseries\itshape\color{green!40!black},
        commentstyle=\bfseries\itshape\color{black!60},
        % Line numbers
        xleftmargin={0.75cm},
        numbers=left,
        stepnumber=1,
        firstnumber=1,
        numberfirstline=true,
        showspaces=false,
        showtabs=false,
        breaklines=true,
        showstringspaces=false,
        tabsize=1,
        emph={
            gpio_config_t, for, uint8_t, TextView, Toast, Button, EditText, ImageView, Typeface, Intent, WebView, WebSettings, SwipeRefreshLayout, RelativeLayout, Animation, AlertDialog, SharedPreferences, Editor, ToggleButton, CardView, LinearLayout, gradient, shape,
        },
        emphstyle={\bfseries\color{pblue}},
        frame=l
    }
}

\begin{document}

\maketitle
\frontmatter

\tableofcontents

\mainmatter

\chapter{ESP32简介}

ESP32-C3 SoC 芯片支持以下功能:

\begin{itemize}
  \item 2.4 GHz Wi-Fi
  \item 低功耗蓝牙
  \item 高性能 32 位 RISC-V 单核处理器
  \item 多种外设
  \item 内置安全硬件
\end{itemize}

ESP32-C3 采用 40 nm 工艺制成,具有最佳的功耗性能、射频性能、稳定性、通用性和可靠性,适用于各种应用场景和不同功耗需求。

此芯片由乐鑫公司开发。

\begin{marker}
  我们使用的芯片是 ESP32-C3 。
\end{marker}

\chapter{安装开发工具ESP-IDF}

ESP-IDF 需要安装一些必备工具,才能围绕 ESP32-C3 构建固件,包括 Python、Git、交叉编译器、CMake 和 Ninja 编译工具等。

在本入门指南中,我们通过 \textcolor{red}{命令行} 进行有关操作。

\begin{marker}
限定条件:

\begin{itemize}
\item 请注意 ESP-IDF 和 ESP-IDF 工具的安装路径不能超过 90 个字符,安装路径过长可能会导致构建失败。
\item Python 或 ESP-IDF 的安装路径中一定不能包含空格或括号。
\item 除非操作系统配置为支持 Unicode UTF-8,否则 Python 或 ESP-IDF 的安装路径中也不能包括特殊字符(非 ASCII 码字符)
\item 各种路径中不要有中文!
\end{itemize}

系统管理员可以通过如下方式将操作系统配置为支持 Unicode UTF-8:控制面板-更改日期、时间或数字格式-管理选项卡-更改系统地域-勾选选项 “Beta:使用 Unicode UTF-8 支持全球语言”-点击确定-重启电脑。
\end{marker}

\section{离线安装ESP-IDF}

点击\href{https://dl.espressif.com/dl/esp-idf/?idf=4.4}{链接}下载离线安装包。

\begin{figure}[!htb]
\centering
\includegraphics[width=0.9\textwidth]{1.png}
\caption{离线安装包}
\end{figure}

\section{安装内容}

安装程序会安装以下组件:

\begin{itemize}
\item 内置的 Python
\item 交叉编译器
\item OpenOCD
\item CMake 和 Ninja 编译工具
\item ESP-IDF
\end{itemize}

安装程序允许将程序下载到现有的ESP-IDF目录。

推荐将ESP-IDF下载到 \Colorbox{lightgrey}{\lstinline{%userprofile%\Desktop\esp-idf}}目录下,其中\Colorbox{lightgrey}{\lstinline}代表家目录。

\section{启动ESP-IDF环境}

安装结束时,如果勾选了 \Colorbox{lightgrey}{\lstinline{Run ESP-IDF PowerShell Environment}} 或 \Colorbox{lightgrey}{\lstinline{Run ESP-IDF Command Prompt (cmd.exe)}},安装程序会在选定的提示符窗口启动 ESP-IDF。

Run ESP-IDF PowerShell Environment:

\begin{figure}[!htb]
\centering
\includegraphics[width=0.9\textwidth]{esp-idf-installer-screenshot-powershell.png}
\caption{PowerShell}
\end{figure}

\chapter{创建工程}

现在,可以准备开发 ESP32 应用程序了。可以从 ESP-IDF 中 examples 目录下的 \Colorbox{lightgrey}{\lstinline{get-started/hello_world}} 工程开始。

\begin{marker}
    ESP-IDF 编译系统不支持 ESP-IDF 路径或其工程路径中带有空格。
\end{marker}

将 \Colorbox{lightgrey}{\lstinline{get-started/hello_world}} 工程复制至本地的 \Colorbox{lightgrey}{\lstinline{~/esp}} 目录下:

\begin{mycode}{复制工程命令}
$ cd %userprofile%\esp
$ xcopy /e /i %IDF_PATH%\examples\get-started\hello_world hello_world
\end{mycode}


\begin{marker}
    ESP-IDF 的 examples 目录下有一系列示例工程,可以按照上述方法复制并运行其中的任何示例,也可以直接编译示例,无需进行复制。
\end{marker}

\section{连接设备}

现在,请将 ESP32 开发板连接到 PC,并查看开发板使用的串口。

在 Windows 操作系统中,串口名称通常以 COM 开头。

\section{配置工程}

请进入 \Colorbox{lightgrey}{\lstinline{hello_world}} 目录,设置 ESP32-C3 为目标芯片,然后运行工程配置工具 menuconfig 。

\begin{mycode}{配置命令}
cd %userprofile%\esp\hello_world
idf.py set-target esp32c3
idf.py menuconfig
\end{mycode}

打开一个新工程后,应首先使用 \Colorbox{lightgrey}{\lstinline{idf.py set-target esp32c3}} 设置“目标”芯片。注意,此操作将清除并初始化项目之前的编译和配置(如有)。也可以直接将“目标”配置为环境变量(此时可跳过该步骤)。

正确操作上述步骤后,系统将显示以下菜单:

\begin{figure}[!htb]
\centering
\includegraphics[width=0.9\textwidth]{project-configuration.png}
\caption{配置界面示意图}
\end{figure}

可以通过此菜单设置项目的具体变量,包括 Wi-Fi 网络名称、密码和处理器速度等。\Colorbox{lightgrey}{\lstinline{hello_world}} 示例项目会以默认配置运行,因此在这一项目中,可以跳过使用 menuconfig 进行项目配置这一步骤。

\section{编译工程}

请使用以下命令,编译烧录工程:

\begin{mycode}{编译工程的命令}
idf.py build
\end{mycode}

运行以上命令可以编译应用程序和所有 ESP-IDF 组件,接着生成引导加载程序、分区表和应用程序二进制文件。

\begin{mycode}{运行示意图}
$ idf.py build
Running cmake in directory /path/to/hello_world/build
Executing "cmake -G Ninja --warn-uninitialized /path/to/hello_world"...
Warn about uninitialized values.
-- Found Git: /usr/bin/git (found version "2.17.0")
-- Building empty aws_iot component due to configuration
-- Component names: ...
-- Component paths: ...

... (more lines of build system output)

[527/527] Generating hello_world.bin
esptool.py v2.3.1

Project build complete. To flash, run this command:
../../../components/esptool_py/esptool/esptool.py -p (PORT) -b 921600 write_flash --flash_mode dio --flash_size detect --flash_freq 40m 0x10000 build/hello_world.bin  build 0x1000 build/bootloader/bootloader.bin 0x8000 build/partition_table/partition-table.bin
or run 'idf.py -p PORT flash'
\end{mycode}

如果一切正常,编译完成后将生成 \Colorbox{lightgrey}{\lstinline{.bin}} 文件。

\section{烧录到设备}

请运行以下命令,将刚刚生成的二进制文件烧录至 ESP32 开发板:

\begin{mycode}{编译加烧录}
idf.py flash
\end{mycode}

\begin{marker}
勾选 flash 选项将自动编译并烧录工程,因此无需再运行 \Colorbox{lightgrey}{\lstinline{idf.py build}}。
\end{marker}

\section{常规操作}

在烧录过程中,会看到类似如下的输出日志:

\begin{mycode}{输出日志}
...
esptool.py --chip esp32 -p /dev/ttyUSB0 -b 460800 --before=default_reset --after=hard_reset write_flash --flash_mode dio --flash_freq 40m --flash_size 2MB 0x8000 partition_table/partition-table.bin 0x1000 bootloader/bootloader.bin 0x10000 hello_world.bin
esptool.py v3.0-dev
Serial port /dev/ttyUSB0
Connecting........_
Chip is ESP32D0WDQ6 (revision 0)
Features: WiFi, BT, Dual Core, Coding Scheme None
Crystal is 40MHz
MAC: 24:0a:c4:05:b9:14
Uploading stub...
Running stub...
Stub running...
Changing baud rate to 460800
Changed.
Configuring flash size...
Compressed 3072 bytes to 103...
Writing at 0x00008000... (100 %)
Wrote 3072 bytes (103 compressed) at 0x00008000 in 0.0 seconds (effective 5962.8 kbit/s)...
Hash of data verified.
Compressed 26096 bytes to 15408...
Writing at 0x00001000... (100 %)
Wrote 26096 bytes (15408 compressed) at 0x00001000 in 0.4 seconds (effective 546.7 kbit/s)...
Hash of data verified.
Compressed 147104 bytes to 77364...
Writing at 0x00010000... (20 %)
Writing at 0x00014000... (40 %)
Writing at 0x00018000... (60 %)
Writing at 0x0001c000... (80 %)
Writing at 0x00020000... (100 %)
Wrote 147104 bytes (77364 compressed) at 0x00010000 in 1.9 seconds (effective 615.5 kbit/s)...
Hash of data verified.

Leaving...
Hard resetting via RTS pin...
Done
\end{mycode}

如果一切顺利,烧录完成后,开发板将会复位,应用程序 \Colorbox{lightgrey}{\lstinline{hello_world}} 开始运行。

\section{监视输出}

使用 \textcolor{red}{串口助手} 监视输出和调试。

\begin{marker}
当要进行烧写时,请关闭串口助手!
\end{marker}

\chapter{GPIO操作}

\begin{mycode}{基本配置}
gpio_config_t io_conf;
// 禁用中断
io_conf.intr_type = GPIO_INTR_DISABLE;
// 设置GPIO为输出模式
io_conf.mode = GPIO_MODE_OUTPUT;
// 设置GPIO PIN引脚为 GPIO1 和 GPIO2
io_conf.pin_bit_mask = ((1ULL << GPIO_NUM_1) | (1ULL << GPIO_NUM_2));
// 禁用下拉模式
io_conf.pull_down_en = 0;
// 开启上拉模式
io_conf.pull_up_en = 1;
// 使用以上配置来配置GPIO
gpio_config(&io_conf);
\end{mycode}

\begin{mycode}{配置中断}
// 上升沿触发中断
io_conf.intr_type = GPIO_INTR_POSEDGE;
// 设置为输入模式
io_conf.mode = GPIO_MODE_INPUT;
// 配置引脚
io_conf.pin_bit_mask = (1ULL << GPIO_NUM_0);
gpio_config(&io_conf);
\end{mycode}

\begin{mycode}{操作GPIO引脚}
// 将GPIO口设置为输入模式
gpio_set_direction(GPIO_NUM_2, GPIO_MODE_INPUT);
// 设置输出模式
gpio_set_direction(GPIO_NUM_2, GPIO_MODE_OUTPUT);
// 输出高低电平
gpio_set_level(GPIO_NUM_1, 1);
gpio_set_level(GPIO_NUM_1, 0);
// 获取GPIO的电平
gpio_get_level(GPIO_NUM_2);
\end{mycode}

有了这些API,我们可以实现 $I^2C$ 协议了。然后就可以实现按键功能了。键盘电路图如下:

\begin{figure}[!htb]
\centering
\includegraphics[width=0.9\textwidth]{keyboard.png}
\caption{键盘电路图}
\end{figure}

为了方便操作,我们先来定义一组宏定义以及声明头文件。

先在 main 文件夹中创建 drivers 文件夹,然后创建文件 \Colorbox{lightgrey}{\lstinline{keyboard_driver.h}} 。文件内容如下:

\begin{mycode}{keyboard\_driver.h}
#ifndef __KEYBOARD_DRIVER_H_
#define __KEYBOARD_DRIVER_H_

#include <inttypes.h>
#include "freertos/FreeRTOS.h"
#include "freertos/task.h"
#include "driver/gpio.h"

#define SC12B_SCL GPIO_NUM_1
#define SC12B_SDA GPIO_NUM_2
#define SC12B_INT GPIO_NUM_0

#define I2C_SDA_IN gpio_set_direction(SC12B_SDA, GPIO_MODE_INPUT)
#define I2C_SDA_OUT gpio_set_direction(SC12B_SDA, GPIO_MODE_OUTPUT)

#define I2C_SCL_H gpio_set_level(SC12B_SCL, 1)
#define I2C_SCL_L gpio_set_level(SC12B_SCL, 0)

#define I2C_SDA_H gpio_set_level(SC12B_SDA, 1)
#define I2C_SDA_L gpio_set_level(SC12B_SDA, 0)

#define I2C_READ_SDA gpio_get_level(SC12B_SDA)

void Delay_ms(uint8_t time);
void I2C_Start(void);
void I2C_Stop(void);
void I2C_Ack(uint8_t x);
uint8_t I2C_Wait_Ack(void);
void I2C_Send_Byte(uint8_t d);
uint8_t I2C_Read_Byte(uint8_t ack);
uint8_t SendByteAndGetNACK(uint8_t data);
uint8_t I2C_Read_Key(void);
uint8_t KEYBOARD_read_key(void);
void KEYBORAD_init(void);

#endif
\end{mycode}

然后实现对应的 \Colorbox{lightgrey}{\lstinline{.c}} 文件。

在 drivers 文件夹中创建 \Colorbox{lightgrey}{\lstinline{keyboard_driver.c}} 文件。内容如下:

\begin{mycode}{keyboard\_driver.c}
#include "keyboard_driver.h"

/// 延时函数,使用 FreeRTOS 的 API 进行包装
void Delay_ms(uint8_t time)
{
    vTaskDelay(time / portTICK_PERIOD_MS);
}

/// 产生起始信号
void I2C_Start(void)
{
    I2C_SDA_OUT; // sda线输出
    I2C_SDA_H;
    I2C_SCL_H;
    Delay_ms(1);
    I2C_SDA_L; // START:when CLK is high,DATA change form high to low
    Delay_ms(1);
    I2C_SCL_L; // 钳住I2C总线,准备发送或接收数据
    Delay_ms(1);
}

/// 产生停止信号
void I2C_Stop(void)
{
    I2C_SCL_L;
    I2C_SDA_OUT; // sda线输出
    I2C_SDA_L;   // STOP:when CLK is high DATA change form low to high
    Delay_ms(1);
    I2C_SCL_H;
    Delay_ms(1);
    I2C_SDA_H; // 发送I2C总线结束信号
}

/// 下发应答
void I2C_Ack(uint8_t x)
{
    I2C_SCL_L;
    I2C_SDA_OUT;
    if (x)
    {
        I2C_SDA_H;
    }
    else
    {
        I2C_SDA_L;
    }
    Delay_ms(1);
    I2C_SCL_H;
    Delay_ms(1);
    I2C_SCL_L;
}

/// 等待应答信号到来,成功返回 0 。
uint8_t I2C_Wait_Ack(void)
{
    uint8_t ucErrTime = 0;
    I2C_SCL_L;
    I2C_SDA_IN; // SDA设置为输入
    Delay_ms(1);
    I2C_SCL_H;
    Delay_ms(1);
    while (I2C_READ_SDA)
    {
        if (ucErrTime++ > 250)
        {
            // I2C_Stop();
            // printf("接受应答失败\n");
            return 1;
        }
    }
    I2C_SCL_L;
    // printf("接受应答成功\n");
    return 0;
}

/// 发送一个字节
void I2C_Send_Byte(uint8_t d)
{
    uint8_t t = 0;
    I2C_SDA_OUT;
    while (8 > t++)
    {
        I2C_SCL_L;
        Delay_ms(1);
        if (d & 0x80)
        {
            I2C_SDA_H;
        }
        else
        {
            I2C_SDA_L;
        }
        Delay_ms(1); // 对TEA5767这三个延时都是必须的
        I2C_SCL_H;
        Delay_ms(1);
        d <<= 1;
    }
}

/// 读 1 个字节
uint8_t I2C_Read_Byte(uint8_t ack)
{
    uint8_t i = 0;
    uint8_t receive = 0;
    I2C_SDA_IN; // SDA设置为输入
    for (i = 0; i < 8; i++)
    {
        I2C_SCL_L;
        Delay_ms(1);
        I2C_SCL_H;
        receive <<= 1;
        if (I2C_READ_SDA)
        {
            receive++;
        }
        Delay_ms(1);
    }
    I2C_Ack(ack); // 发送ACK
    return receive;
}

/// 发送数据并返回应答
uint8_t SendByteAndGetNACK(uint8_t data)
{
    I2C_Send_Byte(data);
    return I2C_Wait_Ack();
}

/// SC12B 简易读取按键值函数(默认直接读取)
/// 此函数只有初始化配置默认的情况下,直接调用,
/// 如果在操作前有写入或者其他读取不能调用默认
uint8_t I2C_Read_Key(void)
{
    I2C_Start();
    if (SendByteAndGetNACK((0x40 << 1) | 0x01))
    {
        I2C_Stop();
        return 0;
    }
    uint8_t i = 0;
    uint8_t k = 0;
    I2C_SDA_IN; // SDA设置为输入
    while (8 > i)
    {
        i++;
        I2C_SCL_L;
        Delay_ms(1);
        I2C_SCL_H;
        if (!k && I2C_READ_SDA)
        {
            k = i;
        }
        Delay_ms(1);
    }
    if (k)
    {
        I2C_Ack(1);
        I2C_Stop();
        return k;
    }
    I2C_Ack(0);
    I2C_SDA_IN; // SDA设置为输入
    while (16 > i)
    {
        i++;
        I2C_SCL_L;
        Delay_ms(1);
        I2C_SCL_H;
        if (!k && I2C_READ_SDA)
        {
            k = i;
        }
        Delay_ms(1);
    }
    I2C_Ack(1);
    I2C_Stop();
    return k;
}

uint8_t KEYBOARD_read_key(void)
{
    uint16_t key = I2C_Read_Key();
    if (key == 4)
    {
        return 1;
    }
    else if (key == 3)
    {
        return 2;
    }
    else if (key == 2)
    {
        return 3;
    }
    else if (key == 7)
    {
        return 4;
    }
    else if (key == 6)
    {
        return 5;
    }
    else if (key == 5)
    {
        return 6;
    }
    else if (key == 10)
    {
        return 7;
    }
    else if (key == 9)
    {
        return 8;
    }
    else if (key == 8)
    {
        return 9;
    }
    else if (key == 1)
    {
        return 0;
    }
    else if (key == 12)
    {
        return '#';
    }
    else if (key == 11)
    {
        return 'M';
    }
    return 255;
}

/// GPIO初始化
void KEYBORAD_init(void)
{
    gpio_config_t io_conf;
    // disable interrupt
    io_conf.intr_type = GPIO_INTR_DISABLE;
    // set as output mode
    io_conf.mode = GPIO_MODE_OUTPUT;
    // bit mask of the pins that you want to set,e.g.SDA
    io_conf.pin_bit_mask = ((1ULL << SC12B_SCL) | (1ULL << SC12B_SDA));
    // disable pull-down mode
    io_conf.pull_down_en = 0;
    // disable pull-up mode
    io_conf.pull_up_en = 1;
    // configure GPIO with the given settings
    gpio_config(&io_conf);

    // 中断
    io_conf.intr_type = GPIO_INTR_POSEDGE;
    io_conf.mode = GPIO_MODE_INPUT;
    io_conf.pin_bit_mask = (1ULL << SC12B_INT);
    gpio_config(&io_conf);
}
\end{mycode}

驱动编写好之后,我们可以在主函数中和电容键盘进行通信了。当按下按键,会产生中断,通过处理中断来识别我们的按键。

在 \Colorbox{lightgrey}{\lstinline{smart-lock.c}} 文件中,主函数是:\Colorbox{lightgrey}{\lstinline{app_main}},ESP-IDF 在编译整个项目的时候,会将 \Colorbox{lightgrey}{\lstinline{app_main}} 注册为一个RTOS任务。无需我们自己编写 main 函数。参见文件中的第 \ref{line:sp} 行。

\begin{mycode}{smark-lock.c}
// 全局变量,用来存储来自 GPIO 的中断事件
static QueueHandle_t gpio_evt_queue = NULL;

static void IRAM_ATTR gpio_isr_handler(void *arg)
{
  uint32_t gpio_num = (uint32_t)arg;
  // 将产生中断的GPIO引脚号入队列。
  xQueueSendFromISR(gpio_evt_queue, &gpio_num, NULL);
}

// 轮询中断事件队列,然后挨个处理
static void process_isr(void *arg)
{
  uint32_t io_num;
  for (;;)
  {
    if (xQueueReceive(gpio_evt_queue, &io_num, portMAX_DELAY))
    {
      if (io_num == 0)
      {
        uint8_t key = KEYBOARD_read_key();
        printf("按下的键:%d\r\n", key);
      }
    }
  }
}

static void ISR_QUEUE_Init(void)
{
  // 创建一个队列来处理来自GPIO的中断事件
  gpio_evt_queue = xQueueCreate(10, sizeof(uint32_t));
  // 开启 process_isr 任务。
  // 这个任务的作用是轮训存储中断事件的队列,将队列中的事件
  // 挨个出队列并进行处理。
  xTaskCreate(process_isr, "process_isr", 2048, NULL, 10, NULL);

  gpio_install_isr_service(0);
  // 将 SC12B_INT 引脚产生的中断交由 gpio_isr_handler 处理。
  // 也就是说一旦 SC12B_INT 产生中断,则调用 gpio_isr_handler 函数。
  gio_isr_handler_add(SC12B_INT, gpio_isr_handler, (void *)SC12B_INT);
}

// 主程序
void (*@\Colorbox{red}{app\_main}@*)(void) (*@\label{line:sp}@*) (*@\tikz[overlay]{\draw[thick,red,->] (0,0.5ex) -- ++(0.5,0) node[right] {入口点函数}}@*)
{
  ISR_QUEUE_Init();
}
\end{mycode}

\begin{figure}[!htb]
\centering
\includegraphics[width=0.9\textwidth]{process_isr.png}
\caption{处理中断的流程}
\end{figure}

最后我们将编写好的代码添加到 \Colorbox{lightgrey}{\lstinline{main}} 文件夹下的 \Colorbox{lightgrey}{\lstinline{CMakeLists.txt}} 文件中。最终我们的项目的文件如下,\Colorbox{lightgrey}{\lstinline{SRCS}} 包含我们编写的 \Colorbox{lightgrey}{\lstinline{.c}} 文件。\Colorbox{lightgrey}{\lstinline{INCLUDE_DIRS}} 包含我们编写的驱动的文件夹。

\begin{mycode}{CMakeLists.txt}
idf_component_register(
    SRCS
        "smart-lock.c"
        "drivers/keyboard_driver.c"
        "drivers/led_driver.c"
        "drivers/bluetooth_driver.c"
        "drivers/wifi_driver.c"
        "drivers/fingerprint_driver.c"
        "drivers/tcp_driver.c"
        "drivers/motor_driver.c"
        "drivers/audio_driver.c"
    INCLUDE_DIRS
        "."
        "./drivers"
)
\end{mycode}

\chapter{红外遥控(RMT)}

\section{简介}

红外遥控(RMT)外设是一个红外发射和接收控制器。其数据格式灵活,可进一步扩展为多功能的通用收发器,发送或接收多种类型的信号。就网络分层而言,RMT 硬件包含物理层和数据链路层。物理层定义通信介质和比特信号的表示方式,数据链路层定义 RMT 帧的格式。RMT 帧的最小数据单元称为 RMT 符号,在驱动程序中以 \Colorbox{lightgrey}{\lstinline{rmt_symbol_word_t}} 表示。

ESP32-C3 的 RMT 外设存在多个通道,每个通道都可以独立配置为发射器或接收器。

RMT 外设通常支持以下场景:

\begin{itemize}
\item 发送或接收红外信号,支持所有红外线协议,如 NEC 协议
\item 生成通用序列
\item 有限或无限次地在硬件控制的循环中发送信号
\item 多通道同时发送
\item 将载波调制到输出信号或从输入信号解调载波
\end{itemize}

\section{RMT符号的内存布局}

RMT硬件定义了自己的数据模式,称为RMT符号。下图展示了一个 RMT 符号的位字段:每个符号由两对两个值组成,每对中的第一个值是一个 15 位的值,表示信号持续时间,以 RMT 滴答计。每对中的第二个值是一个 1 位的值,表示信号的逻辑电平,即高电平或低电平。

\begin{figure}[!htb]
\centering
\includegraphics[width=0.9\textwidth]{RMT-1.png}
\caption{RMT 符号结构(L-信号电平)}
\end{figure}

\section{RMT发射器概述}

RMT 发送通道 (TX Channel) 的数据路径和控制路径如下图所示:

\begin{figure}[!htb]
\centering
\includegraphics[width=0.9\textwidth]{RMT-2.png}
\caption{RMT发射器概述}
\end{figure}

驱动程序将用户数据编码为 RMT 数据格式,随后由 RMT 发射器根据编码生成波形。在将波形发送到 GPIO 管脚前,还可以调制高频载波信号。

\section{RMT接收器概述}

RMT接收通道(RX Channel)的数据路径和控制路径如下图所示:

\begin{figure}[!htb]
\centering
\includegraphics[width=0.9\textwidth]{RMT-3.png}
\caption{RMT接收器概述}
\end{figure}

RMT 接收器可以对输入信号采样,将其转换为 RMT 数据格式,并将数据存储在内存中。还可以向接收器提供输入信号的基本特征,使其识别信号停止条件,并过滤掉信号干扰和噪声。RMT 外设还支持从基准信号中解调出高频载波信号。

\section{补充知识:数字调制}

\begin{figure}[!htb]
\centering
\includegraphics[width=0.9\textwidth]{RMT-4.png}
\caption{数字调制}
\end{figure}

\section{WS2812}

文件夹 \Colorbox{lightgrey}{\lstinline{esp-idf/examples/peripherals/rmt/led_strip}} 是示例代码。修改 RMT 的 GPIO 引脚就可以直接部署运行。

我们的开发板的原理是 ESP32-C3 芯片使用 RMT 模块的功能通过 GPIO 引脚发送波形。而波形是经过编码的 RGB 值。

原理图如下:

\begin{figure}[!htb]
\centering
\includegraphics[width=0.9\textwidth]{ws2812.png}
\caption{LED灯原理图}
\end{figure}

驱动大部分外设来说,几乎是通过 GPIO 的高低电平来处理,而 ws2812 正是需要这样的电平;RMT(远程控制)模块驱动程序可用于发送和接收红外遥控信号。由于RMT灵活性,驱动程序还可用于生成或接收许多其他类型的信号。由一系列脉冲组成的信号由RMT的发射器根据值列表生成。这些值定义脉冲持续时间和二进制级别。发射器还可以提供载波并用提供的脉冲对其进行调制;总的来说它就是一个中间件,就是通过 RMT 模块可以生成解码成包含脉冲持续时间和二进制电平的值的高低电平,从而实现发送和接收我们想要的信号。

关于这个灯珠的资料网上多的是,我总的概述:

\begin{enumerate}
\item 每颗灯珠内置一个驱动芯片,我们只需要和这个驱动芯片通讯就可以达成调光的目的。所以,我们不需要用 PWM 调节。
\item 它的管脚引出有 4 个,2 个是供电用的。还有 2 个是通讯的,DIN是输入,DOUT是输出。以及其是 5V 电压供电。
\item 根据不同的厂商生产不同,驱动的方式有所不一样!下面发送数据顺序是:\Colorbox{lightgrey}{\lstinline{GREEN -- BLUE -- RED}} 。
\end{enumerate}

\begin{figure}[!htb]
\centering
\includegraphics[width=0.9\textwidth]{rmt-rgb.png}
\caption{发送颜色的顺序}
\end{figure}

\section{代码实现}

由于大部分代码都是示例代码。这里只给出新添加的部分,也就是点亮某一个灯的代码。

\begin{mycode}{点灯}
// `led_num` 参数是要点亮的灯的索引。`LED_NUMBERS == 12`,因为我们有 12 个灯。
void light_led(uint8_t led_num)
{
    for (int i = 0; i < 3; i++)
    {
        // 构建 RGB 像素点
        hue = led_num * 360 / LED_NUMBERS;
        // 编码 RGB 值
        led_strip_hsv2rgb(hue, 30, 30, &red, &green, &blue);
        // 发送顺序 GREEN --> BLUE --> RED
        led_strip_pixels[led_num * 3 + 0] = green;
        led_strip_pixels[led_num * 3 + 1] = blue;
        led_strip_pixels[led_num * 3 + 2] = red;
    }

    // 将 RGB 值通过通道发送至 LED 灯。点亮灯。
    ESP_ERROR_CHECK(rmt_transmit(
        led_chan,
        led_encoder,
        led_strip_pixels,
        sizeof(led_strip_pixels),
        &tx_config));
    ESP_ERROR_CHECK(rmt_tx_wait_all_done(led_chan, portMAX_DELAY));

    // 延时 100 毫秒
    vTaskDelay(100 / portTICK_PERIOD_MS);

    // 清空像素矩阵
    memset(led_strip_pixels, 0, sizeof(led_strip_pixels));

    // 再次发送,将灯灭掉。
    ESP_ERROR_CHECK(rmt_transmit(
        led_chan,
        led_encoder,
        led_strip_pixels,
        sizeof(led_strip_pixels),
        &tx_config));
    ESP_ERROR_CHECK(rmt_tx_wait_all_done(led_chan, portMAX_DELAY));
}
\end{mycode}

尝试编写代码调用点灯方法,将灯点亮。

\end{document}